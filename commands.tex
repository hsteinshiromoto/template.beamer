% !TEX encoding = UTF-8 Unicode

% To redeclare a MathOperator: begin
% Not compatible with arXiv !
\makeatletter
\newcommand\RedeclareMathOperator{%
  \@ifstar{\def\rmo@s{m}\rmo@redeclare}{\def\rmo@s{o}\rmo@redeclare}%
}
% this is taken from \renew@command
\newcommand\rmo@redeclare[2]{%
  \begingroup \escapechar\m@ne\xdef\@gtempa{{\string#1}}\endgroup
  \expandafter\@ifundefined\@gtempa
     {\@latex@error{\noexpand#1undefined}\@ehc}%
     \relax
  \expandafter\rmo@declmathop\rmo@s{#1}{#2}}
% This is just \@declmathop without \@ifdefinable
\newcommand\rmo@declmathop[3]{%
  \DeclareRobustCommand{#2}{\qopname\newmcodes@#1{#3}}%
}
\@onlypreamble\RedeclareMathOperator
\makeatother
% end %

% Commands

\newcommand{\dx}{\dot{x}}
\newcommand{\dy}{\dot{y}}
\newcommand{\xp}{x^+}
\newcommand{\lloc}{\mathcal{L}_\mathrm{loc}}
\newcommand{\cl}{\mathtt{cl}}
\newcommand{\dom}{\mathtt{dom}}
\newcommand{\domp}{\mathtt{dom}_{\geq0}}
\newcommand{\im}{\mathtt{im}}
\newcommand{\OSC}{\mathcal{OSC}}
\newcommand{\AC}{\mathcal{AC}}
\newcommand{\supp}{\mathtt{supp}}
\newcommand{\MAF}{\mathtt{MAF}}
\newcommand{\id}{\mathtt{id}}
\newcommand{\conv}{\mathtt{conv}}
\newcommand{\sign}{\mathtt{sign}}
\newcommand{\Liploc}{\cL\mathpzc{ip}_\mathrm{loc}}
\newcommand{\Ploc}{\mathcal{P}_{\mathrm{loc}}}
\renewcommand{\ae}{\mathrm{for\ a.e.\ }}
\RedeclareMathOperator{\Re}{\mathtt{Re}}
\RedeclareMathOperator{\Im}{\mathtt{Im}}
\newcommand*{\qedf}{\hfill\ensuremath{\blacksquare}}
\newcommand*{\qede}{\hfill\ensuremath{\blacksquare}}

\newcommand{\pfrac}[2]{\raisebox{1pt}{\small \ensuremath{#1}}$/$\raisebox{-2pt}{\small \ensuremath{#2}}}

% Commands with tt


% Operators

\DeclareMathOperator{\range}{\mathtt{range}}
\DeclareMathOperator{\wnull}{\mathtt{null}}
\DeclareMathOperator{\diag}{\mathtt{diag}}
\DeclareMathOperator{\dist}{\mathtt{dist}}
\DeclareMathOperator{\diverg}{\mathtt{div}}
\RedeclareMathOperator{\div}{\mathtt{div}}
\DeclareMathOperator{\sat}{\mathtt{sat}}
\DeclareMathOperator{\real}{\mathtt{real}}
\DeclareMathOperator{\rank}{\mathtt{rank}}

\DeclareMathOperator*{\argmin}{\mathtt{argmin}} %Note: the * that follows \DeclareMathOperator sets the underscored option (the argument) underneath the operator like the \lim operator.

\DeclareMathOperator{\interior}{\mathtt{int}}
\DeclareMathOperator{\ext}{\mathtt{ext}}
\DeclareMathOperator{\dir}{\mathrm{dir}}
\DeclareMathOperator{\gph}{\mathrm{gph}}
\DeclareMathOperator{\card}{\mathtt{card}}
\DeclareMathOperator{\diam}{\mathtt{diam}}
\DeclareMathOperator{\co}{\mathtt{co}}
\DeclareMathOperator{\mes}{\mathtt{mes}}
\DeclareMathOperator{\vol}{\mathtt{vol}}
\DeclareMathOperator{\proj}{\mathtt{proj}}
\DeclareMathOperator{\grad}{\mathtt{grad}}
\DeclareMathOperator*{\esssup}{\mathtt{ess\ sup}}

% Colors

\newcommand{\black}{\color{black}}
\newcommand{\blue}{\color{blue}}
\newcommand{\red}{\color{red}}
\newcommand{\new}{\color{black}}
\renewcommand{\stop}{\color{black}}

%\newcommand{\dif}[1]{\dot{\overgroup{\delta #1}}}
%\ifxetex
%	\newcommand{\widedot}[1]{\dot{\overgroup{#1}}}
%\else
%	\renewcommand{\widedot}[1]{\dot{\overgroup{#1}}}
%\fi

\definecolor{white}{HTML}{FFFFFF}
\newcommand{\white}{\color{white}}