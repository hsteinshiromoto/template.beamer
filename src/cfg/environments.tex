% !TEX encoding = UTF-8 Unicode
% !TEX TS-program = xelatex
% !TEX root = ../main.tex

% \usepackage{ntheorem} % Do not use this package
\usepackage{thmtools}

\iftoggle{presentation}{%

% ---
% Block environments												
%	---


% \setbeamertemplate{blocks}[default]
\setbeamertemplate{blocks}[rounded][shadow=true]
%\setbeamerfont{block title}{series=\bfseries}

% Normal block environment settings
\setbeamercolor{structure}{bg=gblue700,fg=white}
\setbeamercolor*{block title}{fg=gblue700,
bg=white}
\setbeamercolor*{block body}{fg=black,
bg=white}

\makeatletter
\pgfdeclareverticalshading{bmb@transition}{200cm}{%
  color(0pt)=(gblue700); color(2pt)=(gblue700)}
\makeatother


% Example block settings
\setbeamercolor*{block title example}{fg=white,
bg=ggreen700}
\setbeamercolor*{block body example}{fg=black,
bg=ggreen100}
\setbeamercolor{example text}{fg=black}

% Alert block settings
\setbeamercolor*{block title alerted}{fg=white,
bg=gred500}
\setbeamercolor*{block body alerted}{fg=black,
bg=gred100}
\setbeamercolor{alerted text}{fg=black}

\newcommand*\redblock{enhanced,colback=gred50,
boxrule=0.4pt,sharp corners,
colframe=gred900,fonttitle=\bfseries}

\newcommand*\postitblock{enhanced,colback=postitcolor,
boxrule=0.4pt,sharp corners,
colframe=postitcolor,fonttitle=\bfseries}

\newcommand*\whiteblock{enhanced,colback=white,
boxrule=0.4pt,sharp corners,
colframe=ggrey300,fonttitle=\bfseries}

\tcbset{enhanced,colback=gblue50,
boxrule=0.4pt,sharp corners,
colframe=gblue700,fonttitle=\bfseries}

%% Testing blocks

\newtcolorbox{dblock}[2][]{
  beamer,
  colback=white,
  colframe=white,
  coltitle=gblue700,
  fonttitle=\normalfont\large,
  overlay={
 %   \draw[gblue700, thick] 
    \draw[gblue700, ultra thick] 
    ([xshift=6pt,yshift=-0.5\pgflinewidth]title.south west) --
    ([xshift=-6pt,yshift=-0.5\pgflinewidth]title.south east);},
  left=1mm,
  right=1mm,
  top=1mm,
  bottom=1mm,
  middle=1mm,
  fuzzy shadow={0.6ex}{-0.6ex}{1pt}{0.5pt}{black!80!white},
  #1,title=#2}
\makeatletter
\tikzset{beamer@color/.style={top color=white,bottom color=white}}
\makeatother

\newtcolorbox{tblock}[2][]{
  beamer,
  colback=white,
  colframe=white,
  coltitle=ggreen700,
  fonttitle=\normalfont\large,
  overlay={
%    \draw[ggreen700, thick] 
    \draw[ggreen700, ultra thick] 
    ([xshift=6pt,yshift=-0.5\pgflinewidth]title.south west) --
    ([xshift=-6pt,yshift=-0.5\pgflinewidth]title.south east);},
  left=1mm,
  right=1mm,
  top=1mm,
  bottom=1mm,
  middle=1mm,
  fuzzy shadow={0.6ex}{-0.6ex}{1pt}{0.5pt}{black!80!white},
  #1,title=#2}
\makeatletter
\tikzset{beamer@color/.style={top color=white,bottom color=white}}
\makeatother

\newtcolorbox{oblock}[2][]{
  beamer,
  colback=white,
  colframe=white,
  coltitle=gorange700,
  fonttitle=\normalfont\large,
  overlay={
%    \draw[ggreen700, thick] 
    \draw[gorange700, ultra thick] 
    ([xshift=6pt,yshift=-0.5\pgflinewidth]title.south west) --
    ([xshift=-6pt,yshift=-0.5\pgflinewidth]title.south east);},
  left=1mm,
  right=1mm,
  top=1mm,
  bottom=1mm,
  middle=1mm,
  fuzzy shadow={0.6ex}{-0.6ex}{1pt}{0.5pt}{black!80!white},
  #1,title=#2}
\makeatletter
\tikzset{beamer@color/.style={top color=white,bottom color=white}}
\makeatother

% For custom item with the color of the theorem block: currently not working
\newenvironment{tlistenv}{\only{\setbeamercolor{local structure}{fg=gorange700}}}{}

\newtcolorbox{rblock}[2][]{
  beamer,
  colback=white,
  colframe=white,
  coltitle=black,
  fonttitle=\normalfont\large,
  overlay={
    \draw[black, ultra thick] 
%    \draw[gblue700, ultra thick] 
    ([xshift=6pt,yshift=-0.5\pgflinewidth]title.south west) --
    ([xshift=-6pt,yshift=-0.5\pgflinewidth]title.south east);},
  left=1mm,
  right=1mm,
  top=1mm,
  bottom=1mm,
  middle=1mm,
  fuzzy shadow={0.6ex}{-0.6ex}{1pt}{0.5pt}{black!80!white},
  #1,title=#2}
\makeatletter
\tikzset{beamer@color/.style={top color=white,bottom color=white}}
\makeatother

}{

\iftoggle{chapter}{%

	\declaretheorem[style=plain,qed=$\lrcorner$,numberwithin=chapter]{theorem}

}{

	\declaretheorem[style=plain,qed=$\lrcorner$,numbered=yes]{theorem}

}


% Theorem environments

\declaretheorem[style=definition,qed=$\lrcorner$,sibling=theorem]{lemma}
\declaretheorem[style=definition,qed=$\lrcorner$,sibling=theorem]{corollary}
\declaretheorem[style=definition,qed=$\lrcorner$,sibling=theorem]{proposition}
\declaretheorem[style=definition,qed=$\lrcorner$,sibling=theorem]{claim}
\declaretheorem[style=definition,qed=$\lrcorner$,sibling=theorem]{conjecture}

% Remark and definition environments
\declaretheorem[style=definition,qed=$\lrcorner$,sibling=theorem]{assumption}
\crefname{assumption}{Assumption}{Assumptions}
\Crefname{assumption}{Assumption}{Assumptions}
\declaretheorem[style=definition,qed=$\lrcorner$,sibling=theorem]{definition}
\declaretheorem[style=definition,qed=$\lrcorner$,sibling=theorem]{remark}
\declaretheorem[style=definition,qed=$\lrcorner$,sibling=theorem]{problem}
\crefname{problem}{Problem}{Problems}
\Crefname{problem}{Problem}{Problems}

\declaretheorem[style=definition,qed=$\lrcorner$,numberwithin=section]{question}
\declaretheorem[style=definition,qed=$\lrcorner$,numbered=no]{answer}

% Example environments

\declaretheorem[style=definition,qed=$\lrcorner$,sibling=theorem]{example}

% Proof environment
\declaretheoremstyle[
  spaceabove=6pt, spacebelow=6pt,
  headfont=\normalfont\slshape,
  notefont=\normalfont\slshape, notebraces={(}{)},
  bodyfont=\normalfont,
  postheadspace=1em,
  qed=$\blacksquare$
]{proofstyle}
\declaretheorem[name=Proof,style=proofstyle,numbered=no]{theproof}

}